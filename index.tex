% Options for packages loaded elsewhere
\PassOptionsToPackage{unicode}{hyperref}
\PassOptionsToPackage{hyphens}{url}
%
\documentclass[
  letterpaper,
]{scrbook}

\usepackage{amsmath,amssymb}
\usepackage[]{libertinus}
\usepackage{iftex}
\ifPDFTeX
  \usepackage[T1]{fontenc}
  \usepackage[utf8]{inputenc}
  \usepackage{textcomp} % provide euro and other symbols
\else % if luatex or xetex
  \usepackage{unicode-math}
  \defaultfontfeatures{Scale=MatchLowercase}
  \defaultfontfeatures[\rmfamily]{Ligatures=TeX,Scale=1}
\fi
% Use upquote if available, for straight quotes in verbatim environments
\IfFileExists{upquote.sty}{\usepackage{upquote}}{}
\IfFileExists{microtype.sty}{% use microtype if available
  \usepackage[]{microtype}
  \UseMicrotypeSet[protrusion]{basicmath} % disable protrusion for tt fonts
}{}
\makeatletter
\@ifundefined{KOMAClassName}{% if non-KOMA class
  \IfFileExists{parskip.sty}{%
    \usepackage{parskip}
  }{% else
    \setlength{\parindent}{0pt}
    \setlength{\parskip}{6pt plus 2pt minus 1pt}}
}{% if KOMA class
  \KOMAoptions{parskip=half}}
\makeatother
\usepackage{xcolor}
\setlength{\emergencystretch}{3em} % prevent overfull lines
\setcounter{secnumdepth}{5}
% Make \paragraph and \subparagraph free-standing
\ifx\paragraph\undefined\else
  \let\oldparagraph\paragraph
  \renewcommand{\paragraph}[1]{\oldparagraph{#1}\mbox{}}
\fi
\ifx\subparagraph\undefined\else
  \let\oldsubparagraph\subparagraph
  \renewcommand{\subparagraph}[1]{\oldsubparagraph{#1}\mbox{}}
\fi


\providecommand{\tightlist}{%
  \setlength{\itemsep}{0pt}\setlength{\parskip}{0pt}}\usepackage{longtable,booktabs,array}
\usepackage{calc} % for calculating minipage widths
% Correct order of tables after \paragraph or \subparagraph
\usepackage{etoolbox}
\makeatletter
\patchcmd\longtable{\par}{\if@noskipsec\mbox{}\fi\par}{}{}
\makeatother
% Allow footnotes in longtable head/foot
\IfFileExists{footnotehyper.sty}{\usepackage{footnotehyper}}{\usepackage{footnote}}
\makesavenoteenv{longtable}
\usepackage{graphicx}
\makeatletter
\def\maxwidth{\ifdim\Gin@nat@width>\linewidth\linewidth\else\Gin@nat@width\fi}
\def\maxheight{\ifdim\Gin@nat@height>\textheight\textheight\else\Gin@nat@height\fi}
\makeatother
% Scale images if necessary, so that they will not overflow the page
% margins by default, and it is still possible to overwrite the defaults
% using explicit options in \includegraphics[width, height, ...]{}
\setkeys{Gin}{width=\maxwidth,height=\maxheight,keepaspectratio}
% Set default figure placement to htbp
\makeatletter
\def\fps@figure{htbp}
\makeatother

\usepackage{booktabs}
\usepackage{longtable}
\usepackage{array}
\usepackage{multirow}
\usepackage{wrapfig}
\usepackage{float}
\usepackage{colortbl}
\usepackage{pdflscape}
\usepackage{tabu}
\usepackage{threeparttable}
\usepackage{threeparttablex}
\usepackage[normalem]{ulem}
\usepackage{makecell}
\usepackage{xcolor}
\makeatletter
\makeatother
\makeatletter
\@ifpackageloaded{bookmark}{}{\usepackage{bookmark}}
\makeatother
\makeatletter
\@ifpackageloaded{caption}{}{\usepackage{caption}}
\AtBeginDocument{%
\ifdefined\contentsname
  \renewcommand*\contentsname{Table of contents}
\else
  \newcommand\contentsname{Table of contents}
\fi
\ifdefined\listfigurename
  \renewcommand*\listfigurename{List of Figures}
\else
  \newcommand\listfigurename{List of Figures}
\fi
\ifdefined\listtablename
  \renewcommand*\listtablename{List of Tables}
\else
  \newcommand\listtablename{List of Tables}
\fi
\ifdefined\figurename
  \renewcommand*\figurename{Figure}
\else
  \newcommand\figurename{Figure}
\fi
\ifdefined\tablename
  \renewcommand*\tablename{Table}
\else
  \newcommand\tablename{Table}
\fi
}
\@ifpackageloaded{float}{}{\usepackage{float}}
\floatstyle{ruled}
\@ifundefined{c@chapter}{\newfloat{codelisting}{h}{lop}}{\newfloat{codelisting}{h}{lop}[chapter]}
\floatname{codelisting}{Listing}
\newcommand*\listoflistings{\listof{codelisting}{List of Listings}}
\makeatother
\makeatletter
\@ifpackageloaded{caption}{}{\usepackage{caption}}
\@ifpackageloaded{subcaption}{}{\usepackage{subcaption}}
\makeatother
\makeatletter
\@ifpackageloaded{tcolorbox}{}{\usepackage[many]{tcolorbox}}
\makeatother
\makeatletter
\@ifundefined{shadecolor}{\definecolor{shadecolor}{rgb}{.97, .97, .97}}
\makeatother
\makeatletter
\makeatother
\ifLuaTeX
  \usepackage{selnolig}  % disable illegal ligatures
\fi
\IfFileExists{bookmark.sty}{\usepackage{bookmark}}{\usepackage{hyperref}}
\IfFileExists{xurl.sty}{\usepackage{xurl}}{} % add URL line breaks if available
\urlstyle{same} % disable monospaced font for URLs
\hypersetup{
  pdftitle={Project Title},
  pdfauthor={Brian},
  hidelinks,
  pdfcreator={LaTeX via pandoc}}

\title{Project Title}
\author{Brian}
\date{6/11/24}

\begin{document}
\frontmatter
\maketitle
\ifdefined\Shaded\renewenvironment{Shaded}{\begin{tcolorbox}[breakable, boxrule=0pt, interior hidden, borderline west={3pt}{0pt}{shadecolor}, enhanced, frame hidden, sharp corners]}{\end{tcolorbox}}\fi

\renewcommand*\contentsname{Table of contents}
{
\setcounter{tocdepth}{1}
\tableofcontents
}
\mainmatter
\bookmarksetup{startatroot}

\hypertarget{preface}{%
\chapter{Preface}\label{preface}}

Lorem Ipsum is simply dummy text of the printing and typesetting
industry. Lorem Ipsum has been the industry's standard dummy text ever
since the 1500s, when an unknown printer took a galley of type and
scrambled it to make a type specimen book. It has survived not only five
centuries, but also the leap into electronic typesetting, remaining
essentially unchanged. It was popularised in the 1960s with the release
of Letraset sheets containing Lorem Ipsum passages, and more recently
with desktop publishing software like Aldus PageMaker including versions
of Lorem Ipsum.

\hypertarget{lorem-ipsum-2}{%
\section{Lorem Ipsum 2}\label{lorem-ipsum-2}}

\textbf{\emph{Lorem Ipsum}}

Lorem Ipsum is simply dummy text of the printing and typesetting
industry. Lorem Ipsum has been the industry's standard dummy text ever
since the 1500s, when an unknown printer took a galley of type and
scrambled it to make a type specimen book. It has survived not only five
centuries, but also the leap into electronic typesetting, remaining
essentially unchanged. It was popularised in the 1960s with the release
of Letraset sheets containing Lorem Ipsum passages, and more recently
with desktop publishing software like Aldus PageMaker including versions
of Lorem Ipsum.

\newpage

\part{analysis.qmd}

\hypertarget{mapping-counts}{%
\chapter{Mapping + Counts}\label{mapping-counts}}

\hypertarget{introduction}{%
\chapter{Introduction}\label{introduction}}

Note: counts filtered to only include counts conducted between
2018-2022. We have all historical counts for permanent counters and
2017-2023 SDC stored in our database.

\hypertarget{count-locations}{%
\section{Count Locations}\label{count-locations}}

Figure~\ref{fig-count_locs} displays where we have permanent (magenta)
and short duration counts (yellow) available for this effort.

\begin{figure}

{\centering 

}

\caption{\label{fig-count_locs}Permanant and Short Duration Count
Locations}

\end{figure}

\hypertarget{permanent-counters}{%
\subsection{Permanent Counters}\label{permanent-counters}}

\hypertarget{tbl-perm_summary}{}
\begin{longtable}[t]{lllll}
\caption{\label{tbl-perm_summary}Perm Summary/Overview (only includes 2018-2022 counts). }\tabularnewline

\toprule
\cellcolor[HTML]{002060}{\textcolor{white}{\textbf{name}}} & \cellcolor[HTML]{002060}{\textcolor{white}{\textbf{install\_date}}} & \cellcolor[HTML]{002060}{\textcolor{white}{\textbf{uninstall\_date}}} & \cellcolor[HTML]{002060}{\textcolor{white}{\textbf{obs\_duration}}} & \cellcolor[HTML]{002060}{\textcolor{white}{\textbf{modes}}}\\
\midrule
2nd Ave PBL & 4/23/16 &  & 4 years 11 mons 30 days 23:00:00 & Bike\\
39th Ave NE Greenway & 1/1/14 & 5/31/18 & 0 years 4 mons 29 days 23:00:00 & Bike\\
BGT & 1/1/14 &  & 4 years 11 mons 30 days 23:00:00 & Bike and Ped\\
Broadway PBL & 1/1/14 & 10/30/21 & 3 years 9 mons 29 days 23:00:00 & Bike\\
Chief Sealth & 1/1/14 & 10/30/21 & 1 year 4 mons 29 days 23:00:00 & Bike\\
\addlinespace
Elliot Bay & 5/11/13 & 12/11/22 & 4 years 11 mons 10 days 23:00:00 & Bike and Ped\\
Fremont Bridge & 10/8/12 &  & 4 years 11 mons 30 days 23:00:00 & Bike\\
MTS or I-90 Trail & 12/18/13 & 12/31/21 & 3 years 11 mons 30 days 23:00:00 & Bike and Ped\\
NW 58th St & 1/1/14 & 7/31/22 & 4 years 6 mons 30 days 23:00:00 & Bike\\
Spokane Bridge & 11/24/13 &  & 4 years 11 mons 30 days 23:00:00 & Bike\\
\bottomrule
\end{longtable}

\hypertarget{consectuive-zeros}{%
\subsection{Consectuive Zeros}\label{consectuive-zeros}}

\hypertarget{tbl-perm_summary_zero}{}
\begin{longtable}[t]{lrlll}
\caption{\label{tbl-perm_summary_zero}QC: Consectuive Zeros (only includes 2018-2022 counts). }\tabularnewline

\toprule
\cellcolor[HTML]{002060}{\textcolor{white}{\textbf{station\_id}}} & \cellcolor[HTML]{002060}{\textcolor{white}{\textbf{year}}} & \cellcolor[HTML]{002060}{\textcolor{white}{\textbf{n\_zero\_flags}}} & \cellcolor[HTML]{002060}{\textcolor{white}{\textbf{n\_records}}} & \cellcolor[HTML]{002060}{\textcolor{white}{\textbf{pct\_flagged}}}\\
\midrule
2nd\_nb & 2018 & 0 & 8,760 & 0.00\%\\
2nd\_nb & 2019 & 0 & 8,760 & 0.00\%\\
2nd\_nb & 2020 & 0 & 8,784 & 0.00\%\\
2nd\_nb & 2021 & 0 & 8,736 & 0.00\%\\
2nd\_nb & 2022 & 12 & 8,760 & 0.14\%\\
\addlinespace
2nd\_sb & 2018 & 0 & 8,760 & 0.00\%\\
2nd\_sb & 2019 & 0 & 8,760 & 0.00\%\\
2nd\_sb & 2020 & 0 & 8,784 & 0.00\%\\
2nd\_sb & 2021 & 0 & 8,736 & 0.00\%\\
2nd\_sb & 2022 & 12 & 8,760 & 0.14\%\\
\addlinespace
39\_nb & 2018 & 18 & 3,600 & 0.50\%\\
39\_sb & 2018 & 18 & 3,600 & 0.50\%\\
58\_eb & 2018 & 0 & 8,760 & 0.00\%\\
58\_eb & 2019 & 293 & 8,088 & 3.62\%\\
58\_eb & 2020 & 327 & 8,040 & 4.07\%\\
\addlinespace
58\_eb & 2021 & 0 & 8,726 & 0.00\%\\
58\_eb & 2022 & 0 & 5,088 & 0.00\%\\
58\_wb & 2018 & 0 & 8,760 & 0.00\%\\
58\_wb & 2019 & 271 & 8,088 & 3.35\%\\
58\_wb & 2020 & 364 & 8,040 & 4.53\%\\
\addlinespace
58\_wb & 2021 & 0 & 8,726 & 0.00\%\\
58\_wb & 2022 & 0 & 5,088 & 0.00\%\\
90\_b\_eb & 2018 & 231 & 8,760 & 2.64\%\\
90\_b\_eb & 2019 & 308 & 8,760 & 3.52\%\\
90\_b\_eb & 2020 & 170 & 8,784 & 1.94\%\\
\addlinespace
90\_b\_eb & 2021 & 862 & 8,760 & 9.84\%\\
90\_b\_wb & 2018 & 231 & 8,760 & 2.64\%\\
90\_b\_wb & 2019 & 338 & 8,760 & 3.86\%\\
90\_b\_wb & 2020 & 170 & 8,784 & 1.94\%\\
90\_b\_wb & 2021 & 1,794 & 8,760 & 20.48\%\\
\addlinespace
90\_p\_eb & 2018 & 231 & 8,760 & 2.64\%\\
90\_p\_eb & 2019 & 3,370 & 8,760 & 38.47\%\\
90\_p\_eb & 2020 & 7,404 & 8,784 & 84.29\%\\
90\_p\_eb & 2021 & 8,760 & 8,760 & 100.00\%\\
90\_p\_wb & 2018 & 231 & 8,760 & 2.64\%\\
\addlinespace
90\_p\_wb & 2019 & 3,370 & 8,760 & 38.47\%\\
90\_p\_wb & 2020 & 7,404 & 8,784 & 84.29\%\\
90\_p\_wb & 2021 & 8,760 & 8,760 & 100.00\%\\
bgt\_b\_nb & 2018 & 2,390 & 8,760 & 27.28\%\\
bgt\_b\_nb & 2019 & 0 & 8,760 & 0.00\%\\
\addlinespace
bgt\_b\_nb & 2020 & 0 & 7,008 & 0.00\%\\
bgt\_b\_nb & 2021 & 0 & 1,464 & 0.00\%\\
bgt\_b\_nb & 2022 & 0 & 8,760 & 0.00\%\\
bgt\_b\_sb & 2018 & 2,378 & 8,760 & 27.15\%\\
bgt\_b\_sb & 2019 & 0 & 8,760 & 0.00\%\\
\addlinespace
bgt\_b\_sb & 2020 & 0 & 7,008 & 0.00\%\\
bgt\_b\_sb & 2021 & 0 & 1,464 & 0.00\%\\
bgt\_b\_sb & 2022 & 0 & 8,760 & 0.00\%\\
bgt\_p\_nb & 2018 & 2,488 & 8,760 & 28.40\%\\
bgt\_p\_nb & 2019 & 0 & 8,760 & 0.00\%\\
\addlinespace
bgt\_p\_nb & 2020 & 363 & 7,008 & 5.18\%\\
bgt\_p\_nb & 2021 & 1,464 & 1,464 & 100.00\%\\
bgt\_p\_nb & 2022 & 8,760 & 8,760 & 100.00\%\\
bgt\_p\_sb & 2018 & 2,486 & 8,760 & 28.38\%\\
bgt\_p\_sb & 2019 & 0 & 8,760 & 0.00\%\\
\addlinespace
bgt\_p\_sb & 2020 & 0 & 7,008 & 0.00\%\\
bgt\_p\_sb & 2021 & 1,464 & 1,464 & 100.00\%\\
bgt\_p\_sb & 2022 & 8,760 & 8,760 & 100.00\%\\
brd\_nb & 2018 & 112 & 8,760 & 1.28\%\\
brd\_nb & 2019 & 0 & 7,344 & 0.00\%\\
\addlinespace
brd\_nb & 2020 & 629 & 5,112 & 12.30\%\\
brd\_nb & 2021 & 1,093 & 7,272 & 15.03\%\\
brd\_sb & 2018 & 216 & 8,760 & 2.47\%\\
brd\_sb & 2019 & 0 & 7,344 & 0.00\%\\
brd\_sb & 2020 & 727 & 5,112 & 14.22\%\\
\addlinespace
brd\_sb & 2021 & 1,155 & 7,272 & 15.88\%\\
cs\_b\_nb & 2020 & 257 & 5,136 & 5.00\%\\
cs\_b\_nb & 2021 & 2,444 & 7,262 & 33.65\%\\
cs\_b\_sb & 2020 & 250 & 5,136 & 4.87\%\\
cs\_b\_sb & 2021 & 2,225 & 7,262 & 30.64\%\\
\addlinespace
cs\_p\_nb & 2020 & 5,136 & 5,136 & 100.00\%\\
cs\_p\_nb & 2021 & 7,262 & 7,262 & 100.00\%\\
cs\_p\_sb & 2020 & 5,136 & 5,136 & 100.00\%\\
cs\_p\_sb & 2021 & 7,262 & 7,262 & 100.00\%\\
eb\_b\_nb & 2018 & 0 & 8,760 & 0.00\%\\
\addlinespace
eb\_b\_nb & 2019 & 0 & 8,760 & 0.00\%\\
eb\_b\_nb & 2020 & 0 & 8,784 & 0.00\%\\
eb\_b\_nb & 2021 & 1,166 & 8,760 & 13.31\%\\
eb\_b\_nb & 2022 & 315 & 8,280 & 3.80\%\\
eb\_b\_sb & 2018 & 0 & 8,760 & 0.00\%\\
\addlinespace
eb\_b\_sb & 2019 & 0 & 8,760 & 0.00\%\\
eb\_b\_sb & 2020 & 0 & 8,784 & 0.00\%\\
eb\_b\_sb & 2021 & 1,166 & 8,760 & 13.31\%\\
eb\_b\_sb & 2022 & 315 & 8,280 & 3.80\%\\
eb\_p\_nb & 2018 & 0 & 8,760 & 0.00\%\\
\addlinespace
eb\_p\_nb & 2019 & 0 & 8,760 & 0.00\%\\
eb\_p\_nb & 2020 & 0 & 8,784 & 0.00\%\\
eb\_p\_nb & 2021 & 1,166 & 8,760 & 13.31\%\\
eb\_p\_nb & 2022 & 315 & 8,280 & 3.80\%\\
eb\_p\_sb & 2018 & 0 & 8,760 & 0.00\%\\
\addlinespace
eb\_p\_sb & 2019 & 0 & 8,760 & 0.00\%\\
eb\_p\_sb & 2020 & 0 & 8,784 & 0.00\%\\
eb\_p\_sb & 2021 & 1,166 & 8,760 & 13.31\%\\
eb\_p\_sb & 2022 & 315 & 8,280 & 3.80\%\\
fmt\_e & 2018 & 0 & 8,760 & 0.00\%\\
\addlinespace
fmt\_e & 2019 & 0 & 8,760 & 0.00\%\\
fmt\_e & 2020 & 0 & 8,784 & 0.00\%\\
fmt\_e & 2021 & 0 & 8,760 & 0.00\%\\
fmt\_e & 2022 & 0 & 8,760 & 0.00\%\\
fmt\_w & 2018 & 0 & 8,760 & 0.00\%\\
\addlinespace
fmt\_w & 2019 & 0 & 8,760 & 0.00\%\\
fmt\_w & 2020 & 0 & 8,784 & 0.00\%\\
fmt\_w & 2021 & 0 & 8,760 & 0.00\%\\
fmt\_w & 2022 & 0 & 8,760 & 0.00\%\\
sb\_eb & 2018 & 0 & 8,760 & 0.00\%\\
\addlinespace
sb\_eb & 2019 & 0 & 8,760 & 0.00\%\\
sb\_eb & 2020 & 0 & 8,784 & 0.00\%\\
sb\_eb & 2021 & 706 & 8,736 & 8.08\%\\
sb\_eb & 2022 & 656 & 10,248 & 6.40\%\\
sp\_wb & 2018 & 0 & 8,760 & 0.00\%\\
\addlinespace
sp\_wb & 2019 & 0 & 8,760 & 0.00\%\\
sp\_wb & 2020 & 0 & 8,784 & 0.00\%\\
sp\_wb & 2021 & 792 & 8,736 & 9.07\%\\
sp\_wb & 2022 & 634 & 10,248 & 6.19\%\\
\bottomrule
\end{longtable}

\hypertarget{short-duration-counts}{%
\subsection{Short Duration Counts}\label{short-duration-counts}}

\hypertarget{tbl-sdc_summary}{}
\begin{longtable}[t]{llrrllll}
\caption{\label{tbl-sdc_summary}SDC Summary/Overview. }\tabularnewline

\toprule
\cellcolor[HTML]{002060}{\textcolor{white}{\textbf{dow}}} & \cellcolor[HTML]{002060}{\textcolor{white}{\textbf{study\_time\_cat}}} & \cellcolor[HTML]{002060}{\textcolor{white}{\textbf{n\_sites}}} & \cellcolor[HTML]{002060}{\textcolor{white}{\textbf{n\_hours}}} & \cellcolor[HTML]{002060}{\textcolor{white}{\textbf{n\_bike}}} & \cellcolor[HTML]{002060}{\textcolor{white}{\textbf{n\_ped}}} & \cellcolor[HTML]{002060}{\textcolor{white}{\textbf{bike\_per\_hr}}} & \cellcolor[HTML]{002060}{\textcolor{white}{\textbf{ped\_per\_hr}}}\\
\midrule
Weekday & 10:00 AM-12:00 PM & 50 & 1786 & 17,949.0 & 338,965.0 & 10.0 & 189.8\\
Weekday & 12:00 PM-2:00 PM & 5 & 24 & 466.0 & 11,662.0 & 19.4 & 485.9\\
Weekday & 5:00 PM-7:00 PM & 49 & 1756 & 62,270.0 & 608,536.0 & 35.5 & 346.5\\
Weekend & 10:00 AM-12:00 PM & 4 & 8 & 84.0 & 1,392.0 & 10.5 & 174.0\\
Weekend & 12:00 PM-2:00 PM & 50 & 1770 & 34,339.0 & 559,484.0 & 19.4 & 316.1\\
\bottomrule
\end{longtable}

\hypertarget{tbl-sdc_summary_year}{}
\begin{longtable}[t]{lrllllrr}
\caption{\label{tbl-sdc_summary_year}SDC Locations. }\tabularnewline

\toprule
\cellcolor[HTML]{002060}{\textcolor{white}{\textbf{site\_descr}}} & \cellcolor[HTML]{002060}{\textcolor{white}{\textbf{n\_hours}}} & \cellcolor[HTML]{002060}{\textcolor{white}{\textbf{n\_bike}}} & \cellcolor[HTML]{002060}{\textcolor{white}{\textbf{n\_ped}}} & \cellcolor[HTML]{002060}{\textcolor{white}{\textbf{bike\_per\_hr}}} & \cellcolor[HTML]{002060}{\textcolor{white}{\textbf{ped\_per\_hr}}} & \cellcolor[HTML]{002060}{\textcolor{white}{\textbf{start\_year}}} & \cellcolor[HTML]{002060}{\textcolor{white}{\textbf{end\_year}}}\\
\midrule
12TH AVE AND E MADISON ST & 108 & 3,698.0 & 60,076.0 & 34.2 & 556.3 & 2018 & 2022\\
12TH AVE NE AND NE 65TH ST & 108 & 1,793.0 & 31,122.0 & 16.6 & 288.2 & 2018 & 2022\\
12TH AVE S AND S JACKSON ST & 108 & 3,505.0 & 49,372.0 & 32.5 & 457.1 & 2018 & 2022\\
15TH AVE NW AND NW MARKET ST & 108 & 253.0 & 44,306.0 & 2.3 & 410.2 & 2018 & 2022\\
15TH AVE W AND BALLARD BR & 108 & 249.0 & 4,046.0 & 2.3 & 37.5 & 2018 & 2022\\
\addlinespace
1ST AVE S AND S JACKSON ST & 108 & 2,347.0 & 39,967.0 & 21.7 & 370.1 & 2018 & 2022\\
1ST AVE S AND S LANDER ST & 108 & 617.0 & 10,721.0 & 5.7 & 99.3 & 2018 & 2022\\
23RD AVE AND E UNION ST & 108 & 1,661.0 & 22,367.0 & 15.4 & 207.1 & 2018 & 2022\\
24TH AVE NW AND NW MARKET ST & 108 & 2,478.0 & 37,370.0 & 22.9 & 346.0 & 2018 & 2022\\
26TH AVE SW AND SW BARTON ST & 108 & 226.0 & 8,234.0 & 2.1 & 76.2 & 2018 & 2022\\
\addlinespace
32ND AVE NW AND NW 54TH ST & 108 & 920.0 & 15,112.0 & 8.5 & 139.9 & 2018 & 2022\\
32ND AVE W AND W MCGRAW ST & 106 & 487.0 & 22,148.0 & 4.6 & 208.9 & 2018 & 2022\\
35TH AVE SW AND SW AVALON WAY & 108 & 2,029.0 & 12,141.0 & 18.8 & 112.4 & 2018 & 2022\\
5TH AVE AND STEWART ST & 108 & 1,552.0 & 82,176.0 & 14.4 & 760.9 & 2018 & 2022\\
5TH AVE NE AND NE NORTHGATE WAY & 108 & 228.0 & 27,124.0 & 2.1 & 251.1 & 2018 & 2022\\
\addlinespace
6TH AVE AND MADISON ST & 108 & 488.0 & 23,366.0 & 4.5 & 216.4 & 2018 & 2022\\
7TH AVE S AND S JACKSON ST & 108 & 2,502.0 & 25,280.0 & 23.2 & 234.1 & 2018 & 2022\\
8TH AVE S AND S CLOVERDALE ST & 108 & 617.0 & 3,700.0 & 5.7 & 34.3 & 2018 & 2022\\
8TH AVE S AND S DEARBORN ST & 108 & 1,477.0 & 5,318.0 & 13.7 & 49.2 & 2018 & 2022\\
9TH AVE N AND MERCER SR ST & 104 & 5,431.0 & 20,670.0 & 52.2 & 198.8 & 2018 & 2022\\
\addlinespace
AIRPORT WAY S AND S VALE ST & 108 & 527.0 & 7,555.0 & 4.9 & 70.0 & 2018 & 2022\\
ALASKAN WAY AND BROAD ST & 108 & 5,776.0 & 32,154.0 & 53.5 & 297.7 & 2018 & 2022\\
ALASKAN WAY AND COLUMBIA ST & 102 & 3,061.0 & 25,817.0 & 30.0 & 253.1 & 2018 & 2022\\
BEACON AVE S AND S LANDER ST & 108 & 1,150.0 & 22,634.0 & 10.6 & 209.6 & 2018 & 2022\\
BOREN AVE AND PINE ST & 108 & 3,914.0 & 43,796.0 & 36.2 & 405.5 & 2018 & 2022\\
\addlinespace
BOSTON ST AND QUEEN ANNE AVE N & 104 & 686.0 & 47,237.0 & 6.6 & 454.2 & 2018 & 2022\\
BROAD ST AND VALLEY ST & 106 & 2,332.0 & 31,691.0 & 22.0 & 299.0 & 2018 & 2022\\
BROADWAY AND E PINE ST & 108 & 5,108.0 & 102,787.0 & 47.3 & 951.7 & 2018 & 2022\\
BROADWAY E AND E OLIVE WAY & 106 & 2,408.0 & 134,418.0 & 22.7 & 1,268.1 & 2018 & 2022\\
BROOKLYN AVE NE AND NE 45TH ST & 108 & 1,133.0 & 53,149.0 & 10.5 & 492.1 & 2018 & 2022\\
\addlinespace
CALIFORNIA AVE SW AND FAUNTLEROY WAY SW & 108 & 950.0 & 14,614.0 & 8.8 & 135.3 & 2018 & 2022\\
CALIFORNIA AVE SW AND SW ALASKA ST & 108 & 802.0 & 54,691.0 & 7.4 & 506.4 & 2018 & 2022\\
DENNY WAY AND DEXTER AVE & 108 & 5,286.0 & 50,152.0 & 48.9 & 464.4 & 2018 & 2022\\
EAST GREEN LAKE DR N AND NE RAVENNA EB BV & 108 & 4,609.0 & 46,378.0 & 42.7 & 429.4 & 2018 & 2022\\
EAST MARGINAL WAY S AND S HANFORD ST & 108 & 5,813.0 & 958.0 & 53.8 & 8.9 & 2018 & 2022\\
\addlinespace
EASTLAKE AVE E AND FUHRMAN AVE E & 108 & 9,226.0 & 13,178.0 & 85.4 & 122.0 & 2018 & 2022\\
FAIRVIEW AVE N AND MERCER ST & 72 & 344.0 & 9,102.0 & 4.8 & 126.4 & 2018 & 2022\\
FAIRVIEW AVE N AND VALLEY ST & 108 & 2,086.0 & 12,224.0 & 19.3 & 113.2 & 2018 & 2022\\
FAUNTLEROY WAY SW AND SW CLOVERDALE ST & 106 & 984.0 & 5,070.0 & 9.3 & 47.8 & 2018 & 2022\\
FREMONT AVE N AND N 34TH ST & 108 & 15,517.0 & 47,402.0 & 143.7 & 438.9 & 2018 & 2022\\
\addlinespace
GREENWOOD AVE N AND N 85TH ST & 108 & 1,026.0 & 25,046.0 & 9.5 & 231.9 & 2018 & 2022\\
LAKE CITY WAY NE AND NE 125TH ST & 108 & 379.0 & 21,253.0 & 3.5 & 196.8 & 2018 & 2022\\
LINDEN AVE N AND N 130TH ST & 110 & 2,881.0 & 6,885.0 & 26.2 & 62.6 & 2018 & 2022\\
M L KING JR WAY S AND RAINIER AVE S & 108 & 197.0 & 8,066.0 & 1.8 & 74.7 & 2018 & 2022\\
M L KING JR WR WAY S AND S ALASKA ST & 108 & 399.0 & 13,837.0 & 3.7 & 128.1 & 2018 & 2022\\
\addlinespace
M L KING JR WR WAY S AND S OTHELLO ST & 108 & 289.0 & 29,229.0 & 2.7 & 270.6 & 2018 & 2022\\
MONTLAKE BLVD NE AND NE PACIFIC ST & 108 & 467.0 & 74,707.0 & 4.3 & 691.7 & 2018 & 2022\\
RAINIER AVE S AND S HENDERSON ST & 108 & 352.0 & 11,694.0 & 3.3 & 108.3 & 2018 & 2022\\
SAND POINT WAY NE AND NE 65TH ST & 108 & 903.0 & 6,909.0 & 8.4 & 64.0 & 2018 & 2022\\
STONE WAY N AND N 45TH ST & 108 & 3,945.0 & 22,790.0 & 36.5 & 211.0 & 2018 & 2022\\
\bottomrule
\end{longtable}

\hypertarget{hin-intro}{%
\chapter{HIN Intro}\label{hin-intro}}

There are many established ways to examine crashes to better understand
traffic safety patterns. Hotspot analyses have long been used to address
high crash locations by retrospectively identifying the greatest
concentrations of reported crashes over a determined period of time.
Hotspot analysis is a valuable method to visualize locations with
historic crash issues, but it is less effective at identifying locations
with latent crash risk factors. In this way, it can be described as
reactive. Additionally, hotspots may be less effective for analyzing
bicyclist safety if crash frequencies are low due to geographic
sparsity, which can exacerbate issues related to regression to the mean.
Conversely, a systemic analysis is effective for identifying roadways
with risk factors for crashes, independent from their crash history. For
example, a wide arterial with a 45-mph posted speed limit, high traffic
volumes, no bike facility, and few trip-attracting land uses may not
have any reported bike crashes. However, the roadway and operational
characteristics of that arterial are associated with higher bicycle
crash risk. The absence of crashes is therefore not a reflection of low
crash risk, but a reflection of lack of exposure that hotspot analyses
cannot adequately convey. Systemic analysis is largely proactive; it
allows planners and engineers to find locations that may warrant safety
improvements before crashes have occurred there.

High injury networks strike a balance between entirely retrospective and
entirely proactive methods. Using spatial patterns of crash history, a
High Injury Network identifies areas on the road network where crashes
have been concentrated in sequence. A stretch of arterial roadway with
crashes occurring at every other intersection might not show up on a
traditional hotspot analysis because no one location has multiple
crashes happening in the same place. However, the pattern of crashes all
along the corridor suggests a larger safety issue. Further, the entire
corridor likely shares similar characteristics that could be addressed
systemically -- even the intersections along the corridor that have not
yet had crashes.

This section describes the development of a statewide High Injury
Network and the results of the related High Injury Network analysis. The
High Injury Network was built from a standard sliding windows analysis,
which measures severity-weighted crash density by mode along segments on
the network.

\hypertarget{intro-and-purpose}{%
\chapter{Intro and Purpose}\label{intro-and-purpose}}

The Bipartisan Infrastructure Law (BIL), passed in 2021, created a new
requirement for state departments of transportation to conduct a
Vulnerable Road User Safety Assessment (VRUSA) every five years.
Anchored in the Safe System Approach (SSA), this assessment must use a
data-driven process to identify high-risk areas and incorporate equity
and demographics into the analysis. Official guidance around the VRUSA
recommends the use of a High Injury Network, predictive analysis, and/or
systemic analysis to identify high-risk areas\footnote{\href{https://highways.dot.gov/sites/fhwa.dot.gov/files/2022-\%2010/VRU\%20Safety\%20Assessment\%20Guidance\%20FINAL_508.pdf}{https://highways.dot.gov/sites/fhwa.dot.gov/files/2022-
  10/VRU\%20Safety\%20Assessment\%20Guidance\%20FINAL\_508.pdf}}.

To improve the safety of vulnerable road users in the state of Minnesota
and partially satisfy the new VRUSA requirements, the Minnesota
Department of Transportation's (MnDOT) Office of Traffic Engineering
(OTE) commissioned a Vulnerable Road User Safety Assessment in 2022,
including the development of a High Injury Network for the state and a
study of bicycling crashes from 2016-2019 in urban and rural areas
within the state. The initial VRUSA built upon a recently-completed
study of pedestrian safety in the state. The resulting VRUSA report
captured trends and risk factors related to crashes involving
bicyclists, pedestrians, and other vulnerable road users using personal
conveyances across the state to direct infrastructure improvements and
safety countermeasures, especially those that reduce crashes that result
in cyclists' serious injury or death. This report builds on the prior
VRUSA and safety work and updates the vulnerable road user safety
analysis through:

\begin{enumerate}
\def\labelenumi{\arabic{enumi})}
\tightlist
\item
  a descriptive safety analysis of more recent crash data
\item
  a Statewide High Injury Network, which was built on xxxxxx bicyclist,
  pedestrian, and other vulnerable road user crashes from 2018-2022
\item
  integration of this updated VRUSA into the 2025-2029 SHSP
\end{enumerate}

This report focuses on vulnerable road user crashes, paralleling the
2021 Minnesota Statewide Pedestrian Safety Analysis\footnote{\url{https://edocs-public.dot.state.mn.us/edocs_public/DMResultSet/download?docId=26158751}}
and the 2022 Minnesota Vulnerable Road User Safety
Assessment\footnote{\url{https://www.dot.state.mn.us/trafficeng/safety/vrusa.html}}.
While bicyclists and pedestrians are different roadway users, use
different infrastructure in many places, and have both overlapping and
distinct safety concerns, both are vulnerable roadway users who are
disproportionately killed and seriously injured in the transportation
system. Often, bicycle and pedestrian countermeasures are planned and
implemented in tandem, and an understanding of bicycle and pedestrian
crash trends needs to inform these processes. Collectively, the 2021
Minnesota Statewide Pedestrian Safety Analysis, the 2022 VRUSA, and the
current VRUSA update (including the development of a High Injury Network
for vulnerable road users) constitute a robust, data-driven process for
identifying higher-risk areas in the transportation system.

This report follows the methodology of the initial Minnesota Vulnerable
Road User Safety Assessment. For the descriptive and systemic analyses,
VRU crashes from 2018-2022 are conflated with roadway and environmental
characteristics to create a dataset for analysis,including variables
about injury severity, lighting, roadway functional classification,
development intensity, Suitability of Pedestrian and Cyclist Environment
(SPACE) scores and related factors, and bicycle infrastructure. Given
data limitations, some of the detailed analysis focuses only on MnDOT's
trunk highway network.

This report also presents a statewide High Injury Network, which uses a
standard sliding window analysis to measure severity-weighted crash
density by mode. The HIN section of the analysis includes all vulnerable
road users: bicyclists, pedestrians, and other personal conveyances.

The rest of the report is structured as follows: First, an overview of
the crash data is presented, followed by descriptive and systemic
analyses. The descriptive analyses present trends among crash and
temporal variables. The systemic analysis presents the High Injury
Network.

\hypertarget{data-overview}{%
\section{Data Overview}\label{data-overview}}

\hypertarget{crash-data}{%
\subsection{Crash Data}\label{crash-data}}

Crash, party, and vehicle data that were provided to the consultant team
include reported crashes from 2018 through 2022 for crashes for all
modes (pedestrians, bicyclists, other - personal conveyances, and
motorists).

All crash data were processed by Safe Streets Research \& Consulting
(``Safe Streets'') and loaded into a Postgres database for additional
analysis using Python, SQL, and R programming languages. The crash,
party, and vehicle tables have a relational structure, which is common
for storing crash data. For every reported crash, there is one crash
record. The party and vehicle tables contain information for all the
primary ``actors'' and their respective ``vehicles'' involved in the
crash and have a many-to-one relationship -- i.e., all relevant party
records are matched via a case identification number to the one crash
record. The party and vehicle tables contain information for each
primary person and their ``vehicle'' such as age, sex, pre-crash action,
injury severity, and vehicle characteristics. This structure is shown in
xxxxx.

Safe Streets processed and restructured the crash data used in this
analysis. New variables were calculated and assigned, and the quality of
the data was assessed through a robust quality control process. All
reported crashes were processed (not just VRU crashes), but only crashes
that involved at least one VRU and at least one motorist are included in
this analysis.

Crashes involving a person using a scooter (e.g., shared e-scooter or
ADA assistive device) are defined in the State of Minnesota as
pedestrian crashes. However, they are coded in MnDOT's crash database as
the unit type ``Other -- Personal Conveyance'' rather than as
``Pedestrian''. The ``Other -- Personal Conveyance'' category also
includes many modes that are not pedestrians, such as farm equipment
(tractor, combine), all-terrain vehicles, snowmobiles, horse and buggy,
and the like. There is no single coded field in the crash database that
differentiates between pedestrians using personal conveyance devices and
these other modes. A targeted effort was conducted to classify these
crashes based on a keyword scan of officer narratives. While we could
reliably differentiate these crashes from farm equipment based on this
procedure, we could not consistently differentiate between mobility
scooters and other devices used by people with mobility impairments and
other types of scooters or pedestrian devices. As stated in the 2019
Pedestrian Safety Analysis, a long-term solution to facilitate routine
analysis of these modes in Minnesota would be to update the crash form
with a field to indicate the type of scooter or device involvement
(e.g., e-scooter, kick-scooter, ADA assistive device, moped scooter) and
retrain officers to utilize the new field to record accurate and
detailed information for more streamlined analysis.

Crashes that met one or more of the following criteria were removed from
the study dataset during the data consolidation process (see cccccc for
the number of crashes that met each criterion; crashes can meet more
than one criterion):

\begin{itemize}
\tightlist
\item
  Motorist-only (non-VRU) -- The research team received a complete crash
  database for the years of interest (2018-2022). Because the scope of
  this project is only to analyze vulnerable road users, crashes that do
  not include a bicyclist, pedestrian, or someone potentially using a
  personal conveyance device are excluded from the analysis.
\item
  Missing coordinates - Crash location GPS coordinates were not
  available.
\item
  Farm Equipment -- The ``unit type'' is coded as ``Other -- Personal
  Conveyance'' and the officer narrative includes the words ``tractor'',
  ``horse'', or ``trailer.''
\item
  Too far away from the street or along a private street - The
  geospatial location of the crash is greater than 300 feet from any
  street or the street was a private roadway.
\item
  The crash occurred in a parking lot -- The location type recorded in
  the crash data is a parking lot.
\item
  Crashes that involved only a bicyclist and no other road users are not
  included in the crash data.
\end{itemize}

\hypertarget{injury-severity-assignment}{%
\section{Injury Severity Assignment}\label{injury-severity-assignment}}

The officer-reported injury severity levels used in this analysis are
specific to the most severely injured (MSI) road user involved in the
crash. This injury severity is different than the reported MSI assigned
to each crash record. In most cases, VRUs are the most severely injured
victim involved in the crash. Using the victim-level severity helps
improve accuracy of summarizing injury severities. It should be noted
that research from the San Francisco Department of Public Health has
documented reporting errors related to mis-coded injury severities,
particularly for suspected serious injuries\footnote{\href{https://www.visionzerosf.org/wp-content/uploads/2021/11/Severe-Injury-Trends\%202011-2020\%20final\%20report.pdf}{https://www.visionzerosf.org/wp-content/uploads/2021/11/Severe-Injury-Trends
  2011-2020 final report.pdf}}, suggesting a need for some fluidity when
discussing minor and serious injuries. This analysis does not have
access to hospital records to verify injury severities stored in the
crash data, so the results in this document reflect the best available
data at the time. For reference, the injury severities recorded in the
crash data and summarized in this analysis are defined as followed:

\begin{itemize}
\tightlist
\item
  K - Fatal: A fatal injury is any injury that results in death within
  30 days after the motor vehicle crash in which the injury occurred. If
  the person did not die at the scene but died within 30 days of the
  motor vehicle crash in which the injury occurred, the injury
  classification should be changed from the injury previously assigned
  to ``Fatal Injury.''
\item
  A -- Suspected Serious Injury: An incapacitating injury is any injury,
  other than a fatal injury, which prevents the injured person from
  walking, driving, or normally continuing the activities the person was
  capable of performing before the injury occurred. Also called
  ``Serious Injury'' or ``Injury A''. This category includes:

  \begin{itemize}
  \tightlist
  \item
    severe lacerations
  \item
    broken or distorted limbs
  \item
    skull or chest injuries
  \item
    abdominal injuries
  \item
    unconsciousness at or when taken from the scene of the crash, or
    unable to leave the crash scene without assistance
  \end{itemize}
\item
  B -- Suspected Minor Injury: A minor injury is any injury that is
  evident at the scene of the crash, other than fatal or serious
  injuries. Also called ``Minor Injury'' or ``Injury B''. Examples
  include:

  \begin{itemize}
  \tightlist
  \item
    lump on the head
  \item
    abrasions
  \item
    bruises
  \item
    minor lacerations (cuts on the skin surface with minimal bleeding
    and no exposure of deeper tissue/muscle)
  \end{itemize}
\item
  C -- Possible Injury: A possible injury is any injury reported or
  claimed which is not a fatal, suspected serious, or suspected minor
  injury. Possible injuries are those that are reported by the person or
  are indicated by their behavior, but no wounds or injuries are readily
  evident. Examples include:

  \begin{itemize}
  \tightlist
  \item
    momentary loss of consciousness
  \item
    claim of injury
  \item
    limping
  \item
    complaint of pain or nausea
  \end{itemize}
\item
  O -- Property Damage Only: Crash where only property is damaged. No
  injuries resulted from the crash.
\end{itemize}

\hypertarget{roadway-and-contextual-data}{%
\section{Roadway and Contextual
Data}\label{roadway-and-contextual-data}}

The crash dataset includes many useful variables for analyzing VRU
safety; however, detailed information about roadway conditions and
nearby land uses is also necessary to provide a more complete
understanding of the context in which crashes occurred and support
future countermeasure selection. A robust data collection and
consolidation process was conducted as part of the 2021 MnDOT Statewide
Pedestrian Safety Analysis. Data from that effort was provided to the
study team for use in this VRU assessment. Data collected during the
Statewide Pedestrian Safety Analysis was re-processed using the same
methods documented in the data collection section of the Pedestrian
Safety Analysis. Please refer to the Statewide Pedestrian Crash
Analysis\footnote{\url{https://edocs-public.dot.state.mn.us/edocs_public/DMResultSet/download?docId=26158751}}
for a detailed summary regarding data usage and limitations.

\appendix
\addcontentsline{toc}{part}{Appendices}

\hypertarget{acronyms}{%
\chapter{Acronyms}\label{acronyms}}

\hypertarget{tbl-acronyms}{}
\begin{longtable}[]{@{}ll@{}}
\caption{\label{tbl-acronyms}Acronyms}\tabularnewline
\toprule()
Acronym & Definition \\
\midrule()
\endfirsthead
\toprule()
Acronym & Definition \\
\midrule()
\endhead
AADT & Annual Average Daily Traffic \\
ADA & Americans with Disabilities Act \\
FHWA & Federal Highway Administration \\
GIS & Geographic Information System \\
HIN & High Injury Network \\
KA & Killed of severely injured \\
MSI & Most severely injured \\
MC & Motorcycle \\
MV & Motor Vehicle \\
PED & Pedestrian \\
PHB & Pedestrian Hybrid Beacon \\
RRFB & Rectangular Rapid Flashing Beacon \\
SSA & Safe System Approach \\
VPD & Vehicles per day \\
VRU & Vulnerable road user \\
VRUSA & Vulnerable Road User Safety Assessment \\
K {[}**K**ABCO{]} & Fatal Injury Severity \\
A {[}K**A**BCO{]} & Suspected Serious Injury \\
C {[}KA**B**CO{]} & Minor Injury \\
C {[}KAB**C**O{]} & Possible Injury \\
O {[}KABC**O**{]} or PDO & Property Damage Only \\
KA or KSI & Killed or Seriously Injured \\
\bottomrule()
\end{longtable}


\backmatter

\end{document}
